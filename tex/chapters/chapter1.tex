\chapter{Introduction}
	\section{The Need For Spin Models}
	The study of statistical mechaincs arose due to the need to connect the physical descriptions
	of large particle systems in the macroscopic and microscopic realms. This is done by bridging 
	thermodynamic quantities (macroscopic) , such as temperature, with microscopic
	observables that fluctuate about some average. This is achieved using statistical methods and 
	probability theory. Applying the methods developed by statistical mechanics to various mathematical 
	models has given physicsts enormous insight into various physical phemoena and gives justification 
	to study these models in detail. 

	We begin by considering an observation made by Pierre Curie in 1895, using a substance 
	whose individual atoms are arranged in a regular crystalline lattice. Furthermore, 
	suppose that each atom in the lattice has a magnetic moment which we call it's spin. In 
	this picture, we also assume that these moments tend to align with their neighbours
	(1, this assumption will be made concerete later) and an external magnetic field $H$. We introduce
	a measurement paramter called the magnetization, which is simply the global average of the spins.

	Varying the external magnetic field, we can observe two distinct behaviours around $H=0$, called
	paramagnetic and ferromagnetic behaviour. In the first case, as $H \rightarrow 0$, the global 
	order (ie: spin alignment) is lost and the magnetization tends to zero. In the ferromagnetic case, 
	the local spin interactions are strong enough that the substance maintains a global non-zero 
	magnetization. The value of this magnetization depends on the direection in which the magnetic field
	approachess $0$, from $\pm H$. Thus, a ferromagnet displays a \textit{spontaneous magnetization} 
	$\pm M$ at $H=0$. This is a discontinuity, which corresponds to a \textit{first order phase transition}

	The observations by Curie also established that a temperature dependent transition can occur in 
	ferromagnetic materials to a paramagnetic regime. This transition occurs at a well-defined 
	temperature called the Curie Temperature. A goal of 20th century physics was to be able to describe
	this phase transition using the framework of statistical mechanics. In 1920, Wilhelm Lenz introduced
	what is now called the Ising Model to help understand that phase transition. The one-dimensional case
	was solved by 

	
	 

	\section{Ising Model}

